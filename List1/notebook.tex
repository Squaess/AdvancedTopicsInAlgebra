
    




    
\documentclass[11pt]{article}

    
    \usepackage[breakable]{tcolorbox}
    \tcbset{nobeforeafter} % prevents tcolorboxes being placing in paragraphs
    \usepackage{float}
    \floatplacement{figure}{H} % forces figures to be placed at the correct location
    
    \usepackage[T1]{fontenc}
    % Nicer default font (+ math font) than Computer Modern for most use cases
    \usepackage{mathpazo}

    % Basic figure setup, for now with no caption control since it's done
    % automatically by Pandoc (which extracts ![](path) syntax from Markdown).
    \usepackage{graphicx}
    % We will generate all images so they have a width \maxwidth. This means
    % that they will get their normal width if they fit onto the page, but
    % are scaled down if they would overflow the margins.
    \makeatletter
    \def\maxwidth{\ifdim\Gin@nat@width>\linewidth\linewidth
    \else\Gin@nat@width\fi}
    \makeatother
    \let\Oldincludegraphics\includegraphics
    % Set max figure width to be 80% of text width, for now hardcoded.
    \renewcommand{\includegraphics}[1]{\Oldincludegraphics[width=.8\maxwidth]{#1}}
    % Ensure that by default, figures have no caption (until we provide a
    % proper Figure object with a Caption API and a way to capture that
    % in the conversion process - todo).
    \usepackage{caption}
    \DeclareCaptionLabelFormat{nolabel}{}
    \captionsetup{labelformat=nolabel}

    \usepackage{adjustbox} % Used to constrain images to a maximum size 
    \usepackage{xcolor} % Allow colors to be defined
    \usepackage{enumerate} % Needed for markdown enumerations to work
    \usepackage{geometry} % Used to adjust the document margins
    \usepackage{amsmath} % Equations
    \usepackage{amssymb} % Equations
    \usepackage{textcomp} % defines textquotesingle
    % Hack from http://tex.stackexchange.com/a/47451/13684:
    \AtBeginDocument{%
        \def\PYZsq{\textquotesingle}% Upright quotes in Pygmentized code
    }
    \usepackage{upquote} % Upright quotes for verbatim code
    \usepackage{eurosym} % defines \euro
    \usepackage[mathletters]{ucs} % Extended unicode (utf-8) support
    \usepackage[utf8x]{inputenc} % Allow utf-8 characters in the tex document
    \usepackage{fancyvrb} % verbatim replacement that allows latex
    \usepackage{grffile} % extends the file name processing of package graphics 
                         % to support a larger range 
    % The hyperref package gives us a pdf with properly built
    % internal navigation ('pdf bookmarks' for the table of contents,
    % internal cross-reference links, web links for URLs, etc.)
    \usepackage{hyperref}
    \usepackage{longtable} % longtable support required by pandoc >1.10
    \usepackage{booktabs}  % table support for pandoc > 1.12.2
    \usepackage[inline]{enumitem} % IRkernel/repr support (it uses the enumerate* environment)
    \usepackage[normalem]{ulem} % ulem is needed to support strikethroughs (\sout)
                                % normalem makes italics be italics, not underlines
    \usepackage{mathrsfs}
    

    
    % Colors for the hyperref package
    \definecolor{urlcolor}{rgb}{0,.145,.698}
    \definecolor{linkcolor}{rgb}{.71,0.21,0.01}
    \definecolor{citecolor}{rgb}{.12,.54,.11}

    % ANSI colors
    \definecolor{ansi-black}{HTML}{3E424D}
    \definecolor{ansi-black-intense}{HTML}{282C36}
    \definecolor{ansi-red}{HTML}{E75C58}
    \definecolor{ansi-red-intense}{HTML}{B22B31}
    \definecolor{ansi-green}{HTML}{00A250}
    \definecolor{ansi-green-intense}{HTML}{007427}
    \definecolor{ansi-yellow}{HTML}{DDB62B}
    \definecolor{ansi-yellow-intense}{HTML}{B27D12}
    \definecolor{ansi-blue}{HTML}{208FFB}
    \definecolor{ansi-blue-intense}{HTML}{0065CA}
    \definecolor{ansi-magenta}{HTML}{D160C4}
    \definecolor{ansi-magenta-intense}{HTML}{A03196}
    \definecolor{ansi-cyan}{HTML}{60C6C8}
    \definecolor{ansi-cyan-intense}{HTML}{258F8F}
    \definecolor{ansi-white}{HTML}{C5C1B4}
    \definecolor{ansi-white-intense}{HTML}{A1A6B2}
    \definecolor{ansi-default-inverse-fg}{HTML}{FFFFFF}
    \definecolor{ansi-default-inverse-bg}{HTML}{000000}

    % commands and environments needed by pandoc snippets
    % extracted from the output of `pandoc -s`
    \providecommand{\tightlist}{%
      \setlength{\itemsep}{0pt}\setlength{\parskip}{0pt}}
    \DefineVerbatimEnvironment{Highlighting}{Verbatim}{commandchars=\\\{\}}
    % Add ',fontsize=\small' for more characters per line
    \newenvironment{Shaded}{}{}
    \newcommand{\KeywordTok}[1]{\textcolor[rgb]{0.00,0.44,0.13}{\textbf{{#1}}}}
    \newcommand{\DataTypeTok}[1]{\textcolor[rgb]{0.56,0.13,0.00}{{#1}}}
    \newcommand{\DecValTok}[1]{\textcolor[rgb]{0.25,0.63,0.44}{{#1}}}
    \newcommand{\BaseNTok}[1]{\textcolor[rgb]{0.25,0.63,0.44}{{#1}}}
    \newcommand{\FloatTok}[1]{\textcolor[rgb]{0.25,0.63,0.44}{{#1}}}
    \newcommand{\CharTok}[1]{\textcolor[rgb]{0.25,0.44,0.63}{{#1}}}
    \newcommand{\StringTok}[1]{\textcolor[rgb]{0.25,0.44,0.63}{{#1}}}
    \newcommand{\CommentTok}[1]{\textcolor[rgb]{0.38,0.63,0.69}{\textit{{#1}}}}
    \newcommand{\OtherTok}[1]{\textcolor[rgb]{0.00,0.44,0.13}{{#1}}}
    \newcommand{\AlertTok}[1]{\textcolor[rgb]{1.00,0.00,0.00}{\textbf{{#1}}}}
    \newcommand{\FunctionTok}[1]{\textcolor[rgb]{0.02,0.16,0.49}{{#1}}}
    \newcommand{\RegionMarkerTok}[1]{{#1}}
    \newcommand{\ErrorTok}[1]{\textcolor[rgb]{1.00,0.00,0.00}{\textbf{{#1}}}}
    \newcommand{\NormalTok}[1]{{#1}}
    
    % Additional commands for more recent versions of Pandoc
    \newcommand{\ConstantTok}[1]{\textcolor[rgb]{0.53,0.00,0.00}{{#1}}}
    \newcommand{\SpecialCharTok}[1]{\textcolor[rgb]{0.25,0.44,0.63}{{#1}}}
    \newcommand{\VerbatimStringTok}[1]{\textcolor[rgb]{0.25,0.44,0.63}{{#1}}}
    \newcommand{\SpecialStringTok}[1]{\textcolor[rgb]{0.73,0.40,0.53}{{#1}}}
    \newcommand{\ImportTok}[1]{{#1}}
    \newcommand{\DocumentationTok}[1]{\textcolor[rgb]{0.73,0.13,0.13}{\textit{{#1}}}}
    \newcommand{\AnnotationTok}[1]{\textcolor[rgb]{0.38,0.63,0.69}{\textbf{\textit{{#1}}}}}
    \newcommand{\CommentVarTok}[1]{\textcolor[rgb]{0.38,0.63,0.69}{\textbf{\textit{{#1}}}}}
    \newcommand{\VariableTok}[1]{\textcolor[rgb]{0.10,0.09,0.49}{{#1}}}
    \newcommand{\ControlFlowTok}[1]{\textcolor[rgb]{0.00,0.44,0.13}{\textbf{{#1}}}}
    \newcommand{\OperatorTok}[1]{\textcolor[rgb]{0.40,0.40,0.40}{{#1}}}
    \newcommand{\BuiltInTok}[1]{{#1}}
    \newcommand{\ExtensionTok}[1]{{#1}}
    \newcommand{\PreprocessorTok}[1]{\textcolor[rgb]{0.74,0.48,0.00}{{#1}}}
    \newcommand{\AttributeTok}[1]{\textcolor[rgb]{0.49,0.56,0.16}{{#1}}}
    \newcommand{\InformationTok}[1]{\textcolor[rgb]{0.38,0.63,0.69}{\textbf{\textit{{#1}}}}}
    \newcommand{\WarningTok}[1]{\textcolor[rgb]{0.38,0.63,0.69}{\textbf{\textit{{#1}}}}}
    
    
    % Define a nice break command that doesn't care if a line doesn't already
    % exist.
    \def\br{\hspace*{\fill} \\* }
    % Math Jax compatibility definitions
    \def\gt{>}
    \def\lt{<}
    \let\Oldtex\TeX
    \let\Oldlatex\LaTeX
    \renewcommand{\TeX}{\textrm{\Oldtex}}
    \renewcommand{\LaTeX}{\textrm{\Oldlatex}}
    % Document parameters
    % Document title
    \title{zad1}
    
    
    
    
    
% Pygments definitions
\makeatletter
\def\PY@reset{\let\PY@it=\relax \let\PY@bf=\relax%
    \let\PY@ul=\relax \let\PY@tc=\relax%
    \let\PY@bc=\relax \let\PY@ff=\relax}
\def\PY@tok#1{\csname PY@tok@#1\endcsname}
\def\PY@toks#1+{\ifx\relax#1\empty\else%
    \PY@tok{#1}\expandafter\PY@toks\fi}
\def\PY@do#1{\PY@bc{\PY@tc{\PY@ul{%
    \PY@it{\PY@bf{\PY@ff{#1}}}}}}}
\def\PY#1#2{\PY@reset\PY@toks#1+\relax+\PY@do{#2}}

\expandafter\def\csname PY@tok@w\endcsname{\def\PY@tc##1{\textcolor[rgb]{0.73,0.73,0.73}{##1}}}
\expandafter\def\csname PY@tok@c\endcsname{\let\PY@it=\textit\def\PY@tc##1{\textcolor[rgb]{0.25,0.50,0.50}{##1}}}
\expandafter\def\csname PY@tok@cp\endcsname{\def\PY@tc##1{\textcolor[rgb]{0.74,0.48,0.00}{##1}}}
\expandafter\def\csname PY@tok@k\endcsname{\let\PY@bf=\textbf\def\PY@tc##1{\textcolor[rgb]{0.00,0.50,0.00}{##1}}}
\expandafter\def\csname PY@tok@kp\endcsname{\def\PY@tc##1{\textcolor[rgb]{0.00,0.50,0.00}{##1}}}
\expandafter\def\csname PY@tok@kt\endcsname{\def\PY@tc##1{\textcolor[rgb]{0.69,0.00,0.25}{##1}}}
\expandafter\def\csname PY@tok@o\endcsname{\def\PY@tc##1{\textcolor[rgb]{0.40,0.40,0.40}{##1}}}
\expandafter\def\csname PY@tok@ow\endcsname{\let\PY@bf=\textbf\def\PY@tc##1{\textcolor[rgb]{0.67,0.13,1.00}{##1}}}
\expandafter\def\csname PY@tok@nb\endcsname{\def\PY@tc##1{\textcolor[rgb]{0.00,0.50,0.00}{##1}}}
\expandafter\def\csname PY@tok@nf\endcsname{\def\PY@tc##1{\textcolor[rgb]{0.00,0.00,1.00}{##1}}}
\expandafter\def\csname PY@tok@nc\endcsname{\let\PY@bf=\textbf\def\PY@tc##1{\textcolor[rgb]{0.00,0.00,1.00}{##1}}}
\expandafter\def\csname PY@tok@nn\endcsname{\let\PY@bf=\textbf\def\PY@tc##1{\textcolor[rgb]{0.00,0.00,1.00}{##1}}}
\expandafter\def\csname PY@tok@ne\endcsname{\let\PY@bf=\textbf\def\PY@tc##1{\textcolor[rgb]{0.82,0.25,0.23}{##1}}}
\expandafter\def\csname PY@tok@nv\endcsname{\def\PY@tc##1{\textcolor[rgb]{0.10,0.09,0.49}{##1}}}
\expandafter\def\csname PY@tok@no\endcsname{\def\PY@tc##1{\textcolor[rgb]{0.53,0.00,0.00}{##1}}}
\expandafter\def\csname PY@tok@nl\endcsname{\def\PY@tc##1{\textcolor[rgb]{0.63,0.63,0.00}{##1}}}
\expandafter\def\csname PY@tok@ni\endcsname{\let\PY@bf=\textbf\def\PY@tc##1{\textcolor[rgb]{0.60,0.60,0.60}{##1}}}
\expandafter\def\csname PY@tok@na\endcsname{\def\PY@tc##1{\textcolor[rgb]{0.49,0.56,0.16}{##1}}}
\expandafter\def\csname PY@tok@nt\endcsname{\let\PY@bf=\textbf\def\PY@tc##1{\textcolor[rgb]{0.00,0.50,0.00}{##1}}}
\expandafter\def\csname PY@tok@nd\endcsname{\def\PY@tc##1{\textcolor[rgb]{0.67,0.13,1.00}{##1}}}
\expandafter\def\csname PY@tok@s\endcsname{\def\PY@tc##1{\textcolor[rgb]{0.73,0.13,0.13}{##1}}}
\expandafter\def\csname PY@tok@sd\endcsname{\let\PY@it=\textit\def\PY@tc##1{\textcolor[rgb]{0.73,0.13,0.13}{##1}}}
\expandafter\def\csname PY@tok@si\endcsname{\let\PY@bf=\textbf\def\PY@tc##1{\textcolor[rgb]{0.73,0.40,0.53}{##1}}}
\expandafter\def\csname PY@tok@se\endcsname{\let\PY@bf=\textbf\def\PY@tc##1{\textcolor[rgb]{0.73,0.40,0.13}{##1}}}
\expandafter\def\csname PY@tok@sr\endcsname{\def\PY@tc##1{\textcolor[rgb]{0.73,0.40,0.53}{##1}}}
\expandafter\def\csname PY@tok@ss\endcsname{\def\PY@tc##1{\textcolor[rgb]{0.10,0.09,0.49}{##1}}}
\expandafter\def\csname PY@tok@sx\endcsname{\def\PY@tc##1{\textcolor[rgb]{0.00,0.50,0.00}{##1}}}
\expandafter\def\csname PY@tok@m\endcsname{\def\PY@tc##1{\textcolor[rgb]{0.40,0.40,0.40}{##1}}}
\expandafter\def\csname PY@tok@gh\endcsname{\let\PY@bf=\textbf\def\PY@tc##1{\textcolor[rgb]{0.00,0.00,0.50}{##1}}}
\expandafter\def\csname PY@tok@gu\endcsname{\let\PY@bf=\textbf\def\PY@tc##1{\textcolor[rgb]{0.50,0.00,0.50}{##1}}}
\expandafter\def\csname PY@tok@gd\endcsname{\def\PY@tc##1{\textcolor[rgb]{0.63,0.00,0.00}{##1}}}
\expandafter\def\csname PY@tok@gi\endcsname{\def\PY@tc##1{\textcolor[rgb]{0.00,0.63,0.00}{##1}}}
\expandafter\def\csname PY@tok@gr\endcsname{\def\PY@tc##1{\textcolor[rgb]{1.00,0.00,0.00}{##1}}}
\expandafter\def\csname PY@tok@ge\endcsname{\let\PY@it=\textit}
\expandafter\def\csname PY@tok@gs\endcsname{\let\PY@bf=\textbf}
\expandafter\def\csname PY@tok@gp\endcsname{\let\PY@bf=\textbf\def\PY@tc##1{\textcolor[rgb]{0.00,0.00,0.50}{##1}}}
\expandafter\def\csname PY@tok@go\endcsname{\def\PY@tc##1{\textcolor[rgb]{0.53,0.53,0.53}{##1}}}
\expandafter\def\csname PY@tok@gt\endcsname{\def\PY@tc##1{\textcolor[rgb]{0.00,0.27,0.87}{##1}}}
\expandafter\def\csname PY@tok@err\endcsname{\def\PY@bc##1{\setlength{\fboxsep}{0pt}\fcolorbox[rgb]{1.00,0.00,0.00}{1,1,1}{\strut ##1}}}
\expandafter\def\csname PY@tok@kc\endcsname{\let\PY@bf=\textbf\def\PY@tc##1{\textcolor[rgb]{0.00,0.50,0.00}{##1}}}
\expandafter\def\csname PY@tok@kd\endcsname{\let\PY@bf=\textbf\def\PY@tc##1{\textcolor[rgb]{0.00,0.50,0.00}{##1}}}
\expandafter\def\csname PY@tok@kn\endcsname{\let\PY@bf=\textbf\def\PY@tc##1{\textcolor[rgb]{0.00,0.50,0.00}{##1}}}
\expandafter\def\csname PY@tok@kr\endcsname{\let\PY@bf=\textbf\def\PY@tc##1{\textcolor[rgb]{0.00,0.50,0.00}{##1}}}
\expandafter\def\csname PY@tok@bp\endcsname{\def\PY@tc##1{\textcolor[rgb]{0.00,0.50,0.00}{##1}}}
\expandafter\def\csname PY@tok@fm\endcsname{\def\PY@tc##1{\textcolor[rgb]{0.00,0.00,1.00}{##1}}}
\expandafter\def\csname PY@tok@vc\endcsname{\def\PY@tc##1{\textcolor[rgb]{0.10,0.09,0.49}{##1}}}
\expandafter\def\csname PY@tok@vg\endcsname{\def\PY@tc##1{\textcolor[rgb]{0.10,0.09,0.49}{##1}}}
\expandafter\def\csname PY@tok@vi\endcsname{\def\PY@tc##1{\textcolor[rgb]{0.10,0.09,0.49}{##1}}}
\expandafter\def\csname PY@tok@vm\endcsname{\def\PY@tc##1{\textcolor[rgb]{0.10,0.09,0.49}{##1}}}
\expandafter\def\csname PY@tok@sa\endcsname{\def\PY@tc##1{\textcolor[rgb]{0.73,0.13,0.13}{##1}}}
\expandafter\def\csname PY@tok@sb\endcsname{\def\PY@tc##1{\textcolor[rgb]{0.73,0.13,0.13}{##1}}}
\expandafter\def\csname PY@tok@sc\endcsname{\def\PY@tc##1{\textcolor[rgb]{0.73,0.13,0.13}{##1}}}
\expandafter\def\csname PY@tok@dl\endcsname{\def\PY@tc##1{\textcolor[rgb]{0.73,0.13,0.13}{##1}}}
\expandafter\def\csname PY@tok@s2\endcsname{\def\PY@tc##1{\textcolor[rgb]{0.73,0.13,0.13}{##1}}}
\expandafter\def\csname PY@tok@sh\endcsname{\def\PY@tc##1{\textcolor[rgb]{0.73,0.13,0.13}{##1}}}
\expandafter\def\csname PY@tok@s1\endcsname{\def\PY@tc##1{\textcolor[rgb]{0.73,0.13,0.13}{##1}}}
\expandafter\def\csname PY@tok@mb\endcsname{\def\PY@tc##1{\textcolor[rgb]{0.40,0.40,0.40}{##1}}}
\expandafter\def\csname PY@tok@mf\endcsname{\def\PY@tc##1{\textcolor[rgb]{0.40,0.40,0.40}{##1}}}
\expandafter\def\csname PY@tok@mh\endcsname{\def\PY@tc##1{\textcolor[rgb]{0.40,0.40,0.40}{##1}}}
\expandafter\def\csname PY@tok@mi\endcsname{\def\PY@tc##1{\textcolor[rgb]{0.40,0.40,0.40}{##1}}}
\expandafter\def\csname PY@tok@il\endcsname{\def\PY@tc##1{\textcolor[rgb]{0.40,0.40,0.40}{##1}}}
\expandafter\def\csname PY@tok@mo\endcsname{\def\PY@tc##1{\textcolor[rgb]{0.40,0.40,0.40}{##1}}}
\expandafter\def\csname PY@tok@ch\endcsname{\let\PY@it=\textit\def\PY@tc##1{\textcolor[rgb]{0.25,0.50,0.50}{##1}}}
\expandafter\def\csname PY@tok@cm\endcsname{\let\PY@it=\textit\def\PY@tc##1{\textcolor[rgb]{0.25,0.50,0.50}{##1}}}
\expandafter\def\csname PY@tok@cpf\endcsname{\let\PY@it=\textit\def\PY@tc##1{\textcolor[rgb]{0.25,0.50,0.50}{##1}}}
\expandafter\def\csname PY@tok@c1\endcsname{\let\PY@it=\textit\def\PY@tc##1{\textcolor[rgb]{0.25,0.50,0.50}{##1}}}
\expandafter\def\csname PY@tok@cs\endcsname{\let\PY@it=\textit\def\PY@tc##1{\textcolor[rgb]{0.25,0.50,0.50}{##1}}}

\def\PYZbs{\char`\\}
\def\PYZus{\char`\_}
\def\PYZob{\char`\{}
\def\PYZcb{\char`\}}
\def\PYZca{\char`\^}
\def\PYZam{\char`\&}
\def\PYZlt{\char`\<}
\def\PYZgt{\char`\>}
\def\PYZsh{\char`\#}
\def\PYZpc{\char`\%}
\def\PYZdl{\char`\$}
\def\PYZhy{\char`\-}
\def\PYZsq{\char`\'}
\def\PYZdq{\char`\"}
\def\PYZti{\char`\~}
% for compatibility with earlier versions
\def\PYZat{@}
\def\PYZlb{[}
\def\PYZrb{]}
\makeatother


    % For linebreaks inside Verbatim environment from package fancyvrb. 
    \makeatletter
        \newbox\Wrappedcontinuationbox 
        \newbox\Wrappedvisiblespacebox 
        \newcommand*\Wrappedvisiblespace {\textcolor{red}{\textvisiblespace}} 
        \newcommand*\Wrappedcontinuationsymbol {\textcolor{red}{\llap{\tiny$\m@th\hookrightarrow$}}} 
        \newcommand*\Wrappedcontinuationindent {3ex } 
        \newcommand*\Wrappedafterbreak {\kern\Wrappedcontinuationindent\copy\Wrappedcontinuationbox} 
        % Take advantage of the already applied Pygments mark-up to insert 
        % potential linebreaks for TeX processing. 
        %        {, <, #, %, $, ' and ": go to next line. 
        %        _, }, ^, &, >, - and ~: stay at end of broken line. 
        % Use of \textquotesingle for straight quote. 
        \newcommand*\Wrappedbreaksatspecials {% 
            \def\PYGZus{\discretionary{\char`\_}{\Wrappedafterbreak}{\char`\_}}% 
            \def\PYGZob{\discretionary{}{\Wrappedafterbreak\char`\{}{\char`\{}}% 
            \def\PYGZcb{\discretionary{\char`\}}{\Wrappedafterbreak}{\char`\}}}% 
            \def\PYGZca{\discretionary{\char`\^}{\Wrappedafterbreak}{\char`\^}}% 
            \def\PYGZam{\discretionary{\char`\&}{\Wrappedafterbreak}{\char`\&}}% 
            \def\PYGZlt{\discretionary{}{\Wrappedafterbreak\char`\<}{\char`\<}}% 
            \def\PYGZgt{\discretionary{\char`\>}{\Wrappedafterbreak}{\char`\>}}% 
            \def\PYGZsh{\discretionary{}{\Wrappedafterbreak\char`\#}{\char`\#}}% 
            \def\PYGZpc{\discretionary{}{\Wrappedafterbreak\char`\%}{\char`\%}}% 
            \def\PYGZdl{\discretionary{}{\Wrappedafterbreak\char`\$}{\char`\$}}% 
            \def\PYGZhy{\discretionary{\char`\-}{\Wrappedafterbreak}{\char`\-}}% 
            \def\PYGZsq{\discretionary{}{\Wrappedafterbreak\textquotesingle}{\textquotesingle}}% 
            \def\PYGZdq{\discretionary{}{\Wrappedafterbreak\char`\"}{\char`\"}}% 
            \def\PYGZti{\discretionary{\char`\~}{\Wrappedafterbreak}{\char`\~}}% 
        } 
        % Some characters . , ; ? ! / are not pygmentized. 
        % This macro makes them "active" and they will insert potential linebreaks 
        \newcommand*\Wrappedbreaksatpunct {% 
            \lccode`\~`\.\lowercase{\def~}{\discretionary{\hbox{\char`\.}}{\Wrappedafterbreak}{\hbox{\char`\.}}}% 
            \lccode`\~`\,\lowercase{\def~}{\discretionary{\hbox{\char`\,}}{\Wrappedafterbreak}{\hbox{\char`\,}}}% 
            \lccode`\~`\;\lowercase{\def~}{\discretionary{\hbox{\char`\;}}{\Wrappedafterbreak}{\hbox{\char`\;}}}% 
            \lccode`\~`\:\lowercase{\def~}{\discretionary{\hbox{\char`\:}}{\Wrappedafterbreak}{\hbox{\char`\:}}}% 
            \lccode`\~`\?\lowercase{\def~}{\discretionary{\hbox{\char`\?}}{\Wrappedafterbreak}{\hbox{\char`\?}}}% 
            \lccode`\~`\!\lowercase{\def~}{\discretionary{\hbox{\char`\!}}{\Wrappedafterbreak}{\hbox{\char`\!}}}% 
            \lccode`\~`\/\lowercase{\def~}{\discretionary{\hbox{\char`\/}}{\Wrappedafterbreak}{\hbox{\char`\/}}}% 
            \catcode`\.\active
            \catcode`\,\active 
            \catcode`\;\active
            \catcode`\:\active
            \catcode`\?\active
            \catcode`\!\active
            \catcode`\/\active 
            \lccode`\~`\~ 	
        }
    \makeatother

    \let\OriginalVerbatim=\Verbatim
    \makeatletter
    \renewcommand{\Verbatim}[1][1]{%
        %\parskip\z@skip
        \sbox\Wrappedcontinuationbox {\Wrappedcontinuationsymbol}%
        \sbox\Wrappedvisiblespacebox {\FV@SetupFont\Wrappedvisiblespace}%
        \def\FancyVerbFormatLine ##1{\hsize\linewidth
            \vtop{\raggedright\hyphenpenalty\z@\exhyphenpenalty\z@
                \doublehyphendemerits\z@\finalhyphendemerits\z@
                \strut ##1\strut}%
        }%
        % If the linebreak is at a space, the latter will be displayed as visible
        % space at end of first line, and a continuation symbol starts next line.
        % Stretch/shrink are however usually zero for typewriter font.
        \def\FV@Space {%
            \nobreak\hskip\z@ plus\fontdimen3\font minus\fontdimen4\font
            \discretionary{\copy\Wrappedvisiblespacebox}{\Wrappedafterbreak}
            {\kern\fontdimen2\font}%
        }%
        
        % Allow breaks at special characters using \PYG... macros.
        \Wrappedbreaksatspecials
        % Breaks at punctuation characters . , ; ? ! and / need catcode=\active 	
        \OriginalVerbatim[#1,codes*=\Wrappedbreaksatpunct]%
    }
    \makeatother

    % Exact colors from NB
    \definecolor{incolor}{HTML}{303F9F}
    \definecolor{outcolor}{HTML}{D84315}
    \definecolor{cellborder}{HTML}{CFCFCF}
    \definecolor{cellbackground}{HTML}{F7F7F7}
    
    % prompt
    \newcommand{\prompt}[4]{
        \llap{{\color{#2}[#3]: #4}}\vspace{-1.25em}
    }
    

    
    % Prevent overflowing lines due to hard-to-break entities
    \sloppy 
    % Setup hyperref package
    \hypersetup{
      breaklinks=true,  % so long urls are correctly broken across lines
      colorlinks=true,
      urlcolor=urlcolor,
      linkcolor=linkcolor,
      citecolor=citecolor,
      }
    % Slightly bigger margins than the latex defaults
    
    \geometry{verbose,tmargin=1in,bmargin=1in,lmargin=1in,rmargin=1in}
    
    

    \begin{document}
    
    
    \maketitle
    
    

    
    \hypertarget{a-library-of-operation-of-addition-scalar-multiplication-dot-product-of-vectors.}{%
\subsection{A library of operation of addition, scalar multiplication,
dot product of
vectors.}\label{a-library-of-operation-of-addition-scalar-multiplication-dot-product-of-vectors.}}

    \begin{tcolorbox}[breakable, size=fbox, boxrule=1pt, pad at break*=1mm,colback=cellbackground, colframe=cellborder]
\prompt{In}{incolor}{1}{\hspace{4pt}}
\begin{Verbatim}[commandchars=\\\{\}]
\PY{n}{vector.addition} \PY{o}{\PYZlt{}\PYZhy{}} \PY{n+nf}{function}\PY{p}{(}\PY{n}{a}\PY{p}{,} \PY{n}{b}\PY{p}{)}\PY{p}{\PYZob{}}
    \PY{n}{a} \PY{o}{+} \PY{n}{b}
\PY{p}{\PYZcb{}}

\PY{n}{vector.scalar.multiplication} \PY{o}{\PYZlt{}\PYZhy{}} \PY{n+nf}{function}\PY{p}{(}\PY{n}{a}\PY{p}{,} \PY{n}{b}\PY{p}{)}\PY{p}{\PYZob{}}
    \PY{n}{a} \PY{o}{*} \PY{n}{b}
\PY{p}{\PYZcb{}}

\PY{n}{vector.dot.product} \PY{o}{\PYZlt{}\PYZhy{}} \PY{n+nf}{function}\PY{p}{(}\PY{n}{a}\PY{p}{,} \PY{n}{b}\PY{p}{)}\PY{p}{\PYZob{}}
    \PY{n}{a} \PY{o}{\PYZpc{}*\PYZpc{}} \PY{n}{b}
\PY{p}{\PYZcb{}}
\end{Verbatim}
\end{tcolorbox}

    \begin{tcolorbox}[breakable, size=fbox, boxrule=1pt, pad at break*=1mm,colback=cellbackground, colframe=cellborder]
\prompt{In}{incolor}{2}{\hspace{4pt}}
\begin{Verbatim}[commandchars=\\\{\}]
\PY{n+nf}{vector.addition}\PY{p}{(}\PY{n+nf}{c}\PY{p}{(}\PY{l+m}{1}\PY{p}{,}\PY{l+m}{1}\PY{p}{,}\PY{l+m}{1}\PY{p}{)}\PY{p}{,} \PY{n+nf}{c}\PY{p}{(}\PY{l+m}{1}\PY{p}{,}\PY{l+m}{1}\PY{p}{,}\PY{l+m}{1}\PY{p}{)}\PY{p}{)}
\end{Verbatim}
\end{tcolorbox}

    \begin{enumerate*}
\item 2
\item 2
\item 2
\end{enumerate*}


    
    \begin{tcolorbox}[breakable, size=fbox, boxrule=1pt, pad at break*=1mm,colback=cellbackground, colframe=cellborder]
\prompt{In}{incolor}{3}{\hspace{4pt}}
\begin{Verbatim}[commandchars=\\\{\}]
\PY{n+nf}{vector.scalar.multiplication}\PY{p}{(}\PY{l+m}{4}\PY{p}{,} \PY{n+nf}{c}\PY{p}{(}\PY{l+m}{1}\PY{p}{,}\PY{l+m}{1}\PY{p}{,}\PY{l+m}{1}\PY{p}{,}\PY{l+m}{1}\PY{p}{)}\PY{p}{)}
\end{Verbatim}
\end{tcolorbox}

    \begin{enumerate*}
\item 4
\item 4
\item 4
\item 4
\end{enumerate*}


    
    \begin{tcolorbox}[breakable, size=fbox, boxrule=1pt, pad at break*=1mm,colback=cellbackground, colframe=cellborder]
\prompt{In}{incolor}{4}{\hspace{4pt}}
\begin{Verbatim}[commandchars=\\\{\}]
\PY{n+nf}{vector.dot.product}\PY{p}{(}\PY{n+nf}{c}\PY{p}{(}\PY{l+m}{1}\PY{p}{,}\PY{l+m}{2}\PY{p}{,}\PY{l+m}{3}\PY{p}{)}\PY{p}{,} \PY{n+nf}{c}\PY{p}{(}\PY{l+m}{1}\PY{p}{,}\PY{l+m}{2}\PY{p}{,}\PY{l+m}{4}\PY{p}{)}\PY{p}{)}
\end{Verbatim}
\end{tcolorbox}

    \begin{tabular}{l}
	 17\\
\end{tabular}


    
    \hypertarget{the-same-but-very-simple}{%
\subsection{The same but very simple}\label{the-same-but-very-simple}}

    \begin{tcolorbox}[breakable, size=fbox, boxrule=1pt, pad at break*=1mm,colback=cellbackground, colframe=cellborder]
\prompt{In}{incolor}{5}{\hspace{4pt}}
\begin{Verbatim}[commandchars=\\\{\}]
\PY{n}{addition} \PY{o}{\PYZlt{}\PYZhy{}} \PY{n+nf}{function}\PY{p}{(}\PY{n}{a}\PY{p}{,} \PY{n}{b}\PY{p}{)}\PY{p}{\PYZob{}}
    \PY{n+nf}{if }\PY{p}{(}\PY{n+nf}{length}\PY{p}{(}\PY{n}{a}\PY{p}{)} \PY{o}{!=} \PY{n+nf}{length}\PY{p}{(}\PY{n}{b}\PY{p}{)}\PY{p}{)} \PY{l+m}{0}
    \PY{n}{my\PYZus{}list} \PY{o}{\PYZlt{}\PYZhy{}} \PY{n+nf}{c}\PY{p}{(}\PY{p}{)}
    \PY{n+nf}{for }\PY{p}{(}\PY{n}{i} \PY{n}{in} \PY{l+m}{1}\PY{o}{:}\PY{n+nf}{length}\PY{p}{(}\PY{n}{a}\PY{p}{)}\PY{p}{)} \PY{p}{\PYZob{}}
        \PY{n}{my\PYZus{}list}\PY{n}{[i}\PY{n}{]} \PY{o}{\PYZlt{}\PYZhy{}} \PY{n}{a}\PY{n}{[i}\PY{n}{]} \PY{o}{+} \PY{n}{b}\PY{n}{[i}\PY{n}{]}
    \PY{p}{\PYZcb{}}
    \PY{n}{my\PYZus{}list}
\PY{p}{\PYZcb{}}
\end{Verbatim}
\end{tcolorbox}

    \begin{tcolorbox}[breakable, size=fbox, boxrule=1pt, pad at break*=1mm,colback=cellbackground, colframe=cellborder]
\prompt{In}{incolor}{6}{\hspace{4pt}}
\begin{Verbatim}[commandchars=\\\{\}]
\PY{n}{scalar.mult} \PY{o}{\PYZlt{}\PYZhy{}} \PY{n+nf}{function}\PY{p}{(}\PY{n}{a}\PY{p}{,} \PY{n}{b}\PY{p}{)} \PY{p}{\PYZob{}}
    \PY{n}{result} \PY{o}{=} \PY{n+nf}{c}\PY{p}{(}\PY{p}{)}
    \PY{n+nf}{for }\PY{p}{(}\PY{n}{i} \PY{n}{in} \PY{l+m}{1}\PY{o}{:}\PY{n+nf}{length}\PY{p}{(}\PY{n}{b}\PY{p}{)}\PY{p}{)} \PY{p}{\PYZob{}}
        \PY{n}{result}\PY{n}{[i}\PY{n}{]} \PY{o}{\PYZlt{}\PYZhy{}} \PY{n}{a} \PY{o}{*} \PY{n}{b}\PY{n}{[i}\PY{n}{]}
    \PY{p}{\PYZcb{}}
    \PY{n}{result}
\PY{p}{\PYZcb{}}
\end{Verbatim}
\end{tcolorbox}

    \begin{tcolorbox}[breakable, size=fbox, boxrule=1pt, pad at break*=1mm,colback=cellbackground, colframe=cellborder]
\prompt{In}{incolor}{7}{\hspace{4pt}}
\begin{Verbatim}[commandchars=\\\{\}]
\PY{n}{dot.product} \PY{o}{\PYZlt{}\PYZhy{}} \PY{n+nf}{function}\PY{p}{(}\PY{n}{a}\PY{p}{,} \PY{n}{b}\PY{p}{)} \PY{p}{\PYZob{}}
    \PY{n}{result} \PY{o}{=} \PY{l+m}{0}
    \PY{n+nf}{for }\PY{p}{(}\PY{n}{i} \PY{n}{in} \PY{l+m}{1}\PY{o}{:}\PY{n+nf}{length}\PY{p}{(}\PY{n}{a}\PY{p}{)}\PY{p}{)} \PY{p}{\PYZob{}}
        \PY{n}{result} \PY{o}{\PYZlt{}\PYZhy{}} \PY{n}{result} \PY{o}{+} \PY{n}{a}\PY{n}{[i}\PY{n}{]} \PY{o}{*} \PY{n}{b}\PY{n}{[i}\PY{n}{]}
    \PY{p}{\PYZcb{}}
    \PY{n}{result}
\PY{p}{\PYZcb{}}
\end{Verbatim}
\end{tcolorbox}

    \begin{tcolorbox}[breakable, size=fbox, boxrule=1pt, pad at break*=1mm,colback=cellbackground, colframe=cellborder]
\prompt{In}{incolor}{8}{\hspace{4pt}}
\begin{Verbatim}[commandchars=\\\{\}]
\PY{n}{c1} \PY{o}{\PYZlt{}\PYZhy{}} \PY{n+nf}{c}\PY{p}{(}\PY{l+m}{1}\PY{p}{,}\PY{l+m}{2}\PY{p}{,}\PY{l+m}{3}\PY{p}{,}\PY{l+m}{4}\PY{p}{,}\PY{l+m}{5}\PY{p}{)}
\PY{n}{c2} \PY{o}{\PYZlt{}\PYZhy{}} \PY{n+nf}{c}\PY{p}{(}\PY{l+m}{1}\PY{p}{,}\PY{l+m}{1}\PY{p}{,}\PY{l+m}{1}\PY{p}{,}\PY{l+m}{1}\PY{p}{,}\PY{l+m}{1}\PY{p}{)}
\PY{n}{x} \PY{o}{\PYZlt{}\PYZhy{}} \PY{n+nf}{addition}\PY{p}{(}\PY{n}{c1}\PY{p}{,}\PY{n}{c2}\PY{p}{)}\PY{p}{;} \PY{n}{x}
\end{Verbatim}
\end{tcolorbox}

    \begin{enumerate*}
\item 2
\item 3
\item 4
\item 5
\item 6
\end{enumerate*}


    
    \begin{tcolorbox}[breakable, size=fbox, boxrule=1pt, pad at break*=1mm,colback=cellbackground, colframe=cellborder]
\prompt{In}{incolor}{9}{\hspace{4pt}}
\begin{Verbatim}[commandchars=\\\{\}]
\PY{n}{x} \PY{o}{\PYZlt{}\PYZhy{}} \PY{n+nf}{scalar.mult}\PY{p}{(}\PY{l+m}{2}\PY{p}{,} \PY{n}{c1}\PY{p}{)}\PY{p}{;} \PY{n}{x}
\end{Verbatim}
\end{tcolorbox}

    \begin{enumerate*}
\item 2
\item 4
\item 6
\item 8
\item 10
\end{enumerate*}


    
    \begin{tcolorbox}[breakable, size=fbox, boxrule=1pt, pad at break*=1mm,colback=cellbackground, colframe=cellborder]
\prompt{In}{incolor}{10}{\hspace{4pt}}
\begin{Verbatim}[commandchars=\\\{\}]
\PY{n}{x} \PY{o}{\PYZlt{}\PYZhy{}} \PY{n+nf}{dot.product}\PY{p}{(}\PY{n}{c1}\PY{p}{,}\PY{n}{c2}\PY{p}{)}\PY{p}{;} \PY{n}{x}
\end{Verbatim}
\end{tcolorbox}

    15

    
    \hypertarget{a-library-of-operation-of-addition-multiplication-transposition-of-matrices.}{%
\subsection{A library of operation of addition, multiplication,
transposition of
matrices.}\label{a-library-of-operation-of-addition-multiplication-transposition-of-matrices.}}

    \begin{tcolorbox}[breakable, size=fbox, boxrule=1pt, pad at break*=1mm,colback=cellbackground, colframe=cellborder]
\prompt{In}{incolor}{11}{\hspace{4pt}}
\begin{Verbatim}[commandchars=\\\{\}]
\PY{n}{m.add} \PY{o}{\PYZlt{}\PYZhy{}} \PY{n+nf}{function}\PY{p}{(}\PY{n}{a}\PY{p}{,} \PY{n}{b}\PY{p}{)}\PY{p}{\PYZob{}}
    \PY{n}{a} \PY{o}{+} \PY{n}{b}
\PY{p}{\PYZcb{}}
\end{Verbatim}
\end{tcolorbox}

    \begin{tcolorbox}[breakable, size=fbox, boxrule=1pt, pad at break*=1mm,colback=cellbackground, colframe=cellborder]
\prompt{In}{incolor}{12}{\hspace{4pt}}
\begin{Verbatim}[commandchars=\\\{\}]
\PY{n}{m.mult} \PY{o}{\PYZlt{}\PYZhy{}} \PY{n+nf}{function}\PY{p}{(}\PY{n}{a}\PY{p}{,} \PY{n}{b}\PY{p}{)}\PY{p}{\PYZob{}}
    \PY{n}{a} \PY{o}{\PYZpc{}*\PYZpc{}} \PY{n}{b}
\PY{p}{\PYZcb{}}
\end{Verbatim}
\end{tcolorbox}

    \begin{tcolorbox}[breakable, size=fbox, boxrule=1pt, pad at break*=1mm,colback=cellbackground, colframe=cellborder]
\prompt{In}{incolor}{13}{\hspace{4pt}}
\begin{Verbatim}[commandchars=\\\{\}]
\PY{n}{m.trans} \PY{o}{\PYZlt{}\PYZhy{}} \PY{n+nf}{function}\PY{p}{(}\PY{n}{a}\PY{p}{)}\PY{p}{\PYZob{}}
    \PY{n+nf}{t}\PY{p}{(}\PY{n}{a}\PY{p}{)}
\PY{p}{\PYZcb{}}
\end{Verbatim}
\end{tcolorbox}

    \hypertarget{the-same-but-simple}{%
\subsection{The same but simple}\label{the-same-but-simple}}

    \begin{tcolorbox}[breakable, size=fbox, boxrule=1pt, pad at break*=1mm,colback=cellbackground, colframe=cellborder]
\prompt{In}{incolor}{14}{\hspace{4pt}}
\begin{Verbatim}[commandchars=\\\{\}]
\PY{n}{matrix.addition} \PY{o}{\PYZlt{}\PYZhy{}} \PY{n+nf}{function}\PY{p}{(}\PY{n}{a}\PY{p}{,} \PY{n}{b}\PY{p}{)}\PY{p}{\PYZob{}}
    \PY{n}{long.c} \PY{o}{\PYZlt{}\PYZhy{}} \PY{n+nf}{c}\PY{p}{(}\PY{p}{)}
    \PY{n}{counter} \PY{o}{\PYZlt{}\PYZhy{}} \PY{l+m}{1}
    \PY{n+nf}{for }\PY{p}{(}\PY{n}{row} \PY{n}{in} \PY{l+m}{1}\PY{o}{:}\PY{n+nf}{nrow}\PY{p}{(}\PY{n}{a}\PY{p}{)}\PY{p}{)} \PY{p}{\PYZob{}}
        \PY{n+nf}{for }\PY{p}{(}\PY{n}{col} \PY{n}{in} \PY{l+m}{1}\PY{o}{:}\PY{n+nf}{ncol}\PY{p}{(}\PY{n}{a}\PY{p}{)}\PY{p}{)} \PY{p}{\PYZob{}}
            \PY{n}{long.c}\PY{n}{[counter}\PY{n}{]} \PY{o}{\PYZlt{}\PYZhy{}} \PY{n}{a}\PY{n}{[row}\PY{p}{,} \PY{n}{col}\PY{n}{]} \PY{o}{+} \PY{n}{b}\PY{n}{[row}\PY{p}{,} \PY{n}{col}\PY{n}{]}
            \PY{n}{counter} \PY{o}{\PYZlt{}\PYZhy{}} \PY{n}{counter} \PY{o}{+} \PY{l+m}{1}
        \PY{p}{\PYZcb{}}
    \PY{p}{\PYZcb{}}
    \PY{n+nf}{matrix}\PY{p}{(}\PY{n}{long.c}\PY{p}{,} \PY{n+nf}{nrow}\PY{p}{(}\PY{n}{a}\PY{p}{)}\PY{p}{,} \PY{n+nf}{ncol}\PY{p}{(}\PY{n}{a}\PY{p}{)}\PY{p}{,} \PY{k+kc}{TRUE}\PY{p}{)}
\PY{p}{\PYZcb{}}
\end{Verbatim}
\end{tcolorbox}

    \begin{tcolorbox}[breakable, size=fbox, boxrule=1pt, pad at break*=1mm,colback=cellbackground, colframe=cellborder]
\prompt{In}{incolor}{15}{\hspace{4pt}}
\begin{Verbatim}[commandchars=\\\{\}]
\PY{n}{matrix.mul} \PY{o}{\PYZlt{}\PYZhy{}} \PY{n+nf}{function}\PY{p}{(}\PY{n}{a}\PY{p}{,} \PY{n}{b}\PY{p}{)}\PY{p}{\PYZob{}}
    \PY{n+nf}{if }\PY{p}{(}\PY{o}{!}\PY{n+nf}{ncol}\PY{p}{(}\PY{n}{a}\PY{p}{)} \PY{o}{==} \PY{n+nf}{nrow}\PY{p}{(}\PY{n}{b}\PY{p}{)}\PY{p}{)} \PY{p}{\PYZob{}}
        \PY{l+m}{0}
    \PY{p}{\PYZcb{}} \PY{n}{else} \PY{p}{\PYZob{}}
        \PY{n}{result} \PY{o}{\PYZlt{}\PYZhy{}} \PY{n+nf}{matrix}\PY{p}{(}\PY{l+m}{0}\PY{p}{,} \PY{n+nf}{nrow}\PY{p}{(}\PY{n}{a}\PY{p}{)}\PY{p}{,} \PY{n+nf}{ncol}\PY{p}{(}\PY{n}{b}\PY{p}{)}\PY{p}{)}

        \PY{n+nf}{for }\PY{p}{(}\PY{n}{row} \PY{n}{in} \PY{l+m}{1}\PY{o}{:}\PY{n+nf}{nrow}\PY{p}{(}\PY{n}{a}\PY{p}{)}\PY{p}{)} \PY{p}{\PYZob{}}

            \PY{n+nf}{for }\PY{p}{(}\PY{n}{b\PYZus{}col} \PY{n}{in} \PY{l+m}{1}\PY{o}{:}\PY{n+nf}{ncol}\PY{p}{(}\PY{n}{b}\PY{p}{)}\PY{p}{)} \PY{p}{\PYZob{}}

                 \PY{n+nf}{for }\PY{p}{(}\PY{n}{col} \PY{n}{in} \PY{l+m}{1}\PY{o}{:}\PY{n+nf}{ncol}\PY{p}{(}\PY{n}{a}\PY{p}{)}\PY{p}{)} \PY{p}{\PYZob{}}
                     \PY{n}{result}\PY{n}{[row}\PY{p}{,} \PY{n}{b\PYZus{}col}\PY{n}{]} \PY{o}{\PYZlt{}\PYZhy{}} \PY{n}{result}\PY{n}{[row}\PY{p}{,} \PY{n}{b\PYZus{}col}\PY{n}{]} \PY{o}{+} \PY{p}{(}\PY{n}{a}\PY{n}{[row}\PY{p}{,} \PY{n}{col}\PY{n}{]} \PY{o}{*} \PY{n}{b}\PY{n}{[col}\PY{p}{,} \PY{n}{b\PYZus{}col}\PY{n}{]}\PY{p}{)}
                \PY{p}{\PYZcb{}}

            \PY{p}{\PYZcb{}}

        \PY{p}{\PYZcb{}}
        \PY{n}{result}
    \PY{p}{\PYZcb{}}
\PY{p}{\PYZcb{}}
\end{Verbatim}
\end{tcolorbox}

    \begin{tcolorbox}[breakable, size=fbox, boxrule=1pt, pad at break*=1mm,colback=cellbackground, colframe=cellborder]
\prompt{In}{incolor}{16}{\hspace{4pt}}
\begin{Verbatim}[commandchars=\\\{\}]
\PY{n}{matrix.transpose} \PY{o}{\PYZlt{}\PYZhy{}} \PY{n+nf}{function}\PY{p}{(}\PY{n}{a}\PY{p}{)}\PY{p}{\PYZob{}}
    \PY{n}{result} \PY{o}{=} \PY{n+nf}{matrix}\PY{p}{(}\PY{l+m}{0}\PY{p}{,} \PY{n+nf}{ncol}\PY{p}{(}\PY{n}{a}\PY{p}{)}\PY{p}{,} \PY{n+nf}{nrow}\PY{p}{(}\PY{n}{a}\PY{p}{)}\PY{p}{)}
    \PY{n+nf}{for }\PY{p}{(}\PY{n}{col} \PY{n}{in} \PY{l+m}{1}\PY{o}{:}\PY{n+nf}{ncol}\PY{p}{(}\PY{n}{a}\PY{p}{)}\PY{p}{)}\PY{p}{\PYZob{}}
        \PY{n}{result}\PY{n}{[col}\PY{p}{,}\PY{n}{]} \PY{o}{=} \PY{n}{a}\PY{n}{[}\PY{p}{,}\PY{n}{col}\PY{n}{]}
    \PY{p}{\PYZcb{}}
    \PY{n}{result}   
\PY{p}{\PYZcb{}}
\end{Verbatim}
\end{tcolorbox}

    \begin{tcolorbox}[breakable, size=fbox, boxrule=1pt, pad at break*=1mm,colback=cellbackground, colframe=cellborder]
\prompt{In}{incolor}{17}{\hspace{4pt}}
\begin{Verbatim}[commandchars=\\\{\}]
\PY{n}{m1} \PY{o}{\PYZlt{}\PYZhy{}} \PY{n+nf}{matrix}\PY{p}{(}\PY{n+nf}{c}\PY{p}{(}\PY{l+m}{1}\PY{p}{,}\PY{l+m}{2}\PY{p}{,}\PY{l+m}{3}\PY{p}{,}\PY{l+m}{4}\PY{p}{,}\PY{l+m}{5}\PY{p}{,}\PY{l+m}{6}\PY{p}{)}\PY{p}{,} \PY{l+m}{3}\PY{p}{)}\PY{p}{;} \PY{n}{m1}
\PY{n}{m2} \PY{o}{\PYZlt{}\PYZhy{}} \PY{n+nf}{matrix}\PY{p}{(}\PY{n+nf}{c}\PY{p}{(}\PY{l+m}{1}\PY{p}{,}\PY{l+m}{2}\PY{p}{,}\PY{l+m}{3}\PY{p}{,}\PY{l+m}{4}\PY{p}{,}\PY{l+m}{5}\PY{p}{,}\PY{l+m}{6}\PY{p}{)}\PY{p}{,} \PY{l+m}{3}\PY{p}{)}\PY{p}{;}

\PY{n}{x} \PY{o}{\PYZlt{}\PYZhy{}} \PY{n+nf}{matrix.addition}\PY{p}{(}\PY{n}{m1}\PY{p}{,} \PY{n}{m2}\PY{p}{)}\PY{p}{;} \PY{n}{x}
\PY{n+nf}{matrix.addition}\PY{p}{(}\PY{n}{m1}\PY{p}{,} \PY{n}{m2}\PY{p}{)} \PY{o}{==} \PY{n+nf}{m.add}\PY{p}{(}\PY{n}{m1}\PY{p}{,} \PY{n}{m2}\PY{p}{)}
    
\PY{n}{x} \PY{o}{\PYZlt{}\PYZhy{}} \PY{n+nf}{matrix.mul}\PY{p}{(}\PY{n}{m1}\PY{p}{,} \PY{n+nf}{matrix.transpose}\PY{p}{(}\PY{n}{m2}\PY{p}{)}\PY{p}{)}\PY{p}{;} \PY{n}{x}
\PY{n+nf}{matrix.mul}\PY{p}{(}\PY{n}{m1}\PY{p}{,} \PY{n+nf}{matrix.transpose}\PY{p}{(}\PY{n}{m2}\PY{p}{)}\PY{p}{)} \PY{o}{==} \PY{n+nf}{m.mult}\PY{p}{(}\PY{n}{m1}\PY{p}{,} \PY{n+nf}{m.trans}\PY{p}{(}\PY{n}{m2}\PY{p}{)}\PY{p}{)}
\end{Verbatim}
\end{tcolorbox}

    \begin{tabular}{ll}
	 1 & 4\\
	 2 & 5\\
	 3 & 6\\
\end{tabular}


    
    \begin{tabular}{ll}
	 2  &  8\\
	 4  & 10\\
	 6  & 12\\
\end{tabular}


    
    \begin{tabular}{ll}
	 TRUE & TRUE\\
	 TRUE & TRUE\\
	 TRUE & TRUE\\
\end{tabular}


    
    \begin{tabular}{lll}
	 17 & 22 & 27\\
	 22 & 29 & 36\\
	 27 & 36 & 45\\
\end{tabular}


    
    \begin{tabular}{lll}
	 TRUE & TRUE & TRUE\\
	 TRUE & TRUE & TRUE\\
	 TRUE & TRUE & TRUE\\
\end{tabular}


    
    \hypertarget{matricies-for-tests-to-gaussian-elimination}{%
\subsection{Matricies for tests to gaussian
elimination}\label{matricies-for-tests-to-gaussian-elimination}}

    \begin{tcolorbox}[breakable, size=fbox, boxrule=1pt, pad at break*=1mm,colback=cellbackground, colframe=cellborder]
\prompt{In}{incolor}{18}{\hspace{4pt}}
\begin{Verbatim}[commandchars=\\\{\}]
\PY{n}{m} \PY{o}{\PYZlt{}\PYZhy{}} \PY{n+nf}{matrix}\PY{p}{(}\PY{n+nf}{c}\PY{p}{(}\PY{l+m}{3}\PY{p}{,}\PY{l+m}{1}\PY{p}{,}\PY{l+m}{2}\PY{p}{,}\PY{l+m}{6}\PY{p}{,}\PY{l+m}{4}\PY{p}{,}\PY{l+m}{4}\PY{p}{,} \PY{l+m}{3}\PY{p}{,}\PY{l+m}{4}\PY{p}{,}\PY{l+m}{7}\PY{p}{)}\PY{p}{,} \PY{l+m}{3}\PY{p}{,} \PY{l+m}{3}\PY{p}{)}\PY{p}{;}
\PY{n}{m1} \PY{o}{\PYZlt{}\PYZhy{}} \PY{n+nf}{matrix}\PY{p}{(}\PY{n+nf}{c}\PY{p}{(}\PY{l+m}{3}\PY{p}{,}\PY{l+m}{1}\PY{p}{,}\PY{l+m}{2}\PY{p}{,}\PY{l+m}{6}\PY{p}{,}\PY{l+m}{4}\PY{p}{,}\PY{l+m}{4}\PY{p}{,}\PY{l+m}{3}\PY{p}{,}\PY{l+m}{4}\PY{p}{,}\PY{l+m}{7}\PY{p}{,}\PY{l+m}{1}\PY{p}{,}\PY{l+m}{2}\PY{p}{,}\PY{l+m}{3}\PY{p}{)}\PY{p}{,} \PY{l+m}{3}\PY{p}{,} \PY{l+m}{4}\PY{p}{)}\PY{p}{;}
\PY{n}{m2} \PY{o}{\PYZlt{}\PYZhy{}} \PY{n+nf}{matrix}\PY{p}{(}\PY{n+nf}{c}\PY{p}{(}\PY{l+m}{3}\PY{p}{,}\PY{l+m}{1}\PY{p}{,}\PY{l+m}{2}\PY{p}{,}\PY{l+m}{6}\PY{p}{,}\PY{l+m}{4}\PY{p}{,}\PY{l+m}{4}\PY{p}{,}\PY{l+m}{3}\PY{p}{,}\PY{l+m}{4}\PY{p}{,}\PY{l+m}{7}\PY{p}{,}\PY{l+m}{1}\PY{p}{,}\PY{l+m}{2}\PY{p}{,}\PY{l+m}{3}\PY{p}{)}\PY{p}{,} \PY{l+m}{4}\PY{p}{,} \PY{l+m}{3}\PY{p}{)}\PY{p}{;}
\PY{n}{m.zeros} \PY{o}{\PYZlt{}\PYZhy{}} \PY{n+nf}{matrix}\PY{p}{(}\PY{n+nf}{c}\PY{p}{(}\PY{l+m}{0}\PY{p}{,}\PY{l+m}{1}\PY{p}{,}\PY{l+m}{2}\PY{p}{,}\PY{l+m}{1}\PY{p}{,}\PY{l+m}{0}\PY{p}{,}\PY{l+m}{2}\PY{p}{,}\PY{l+m}{1}\PY{p}{,}\PY{l+m}{2}\PY{p}{,}\PY{l+m}{0}\PY{p}{)}\PY{p}{,} \PY{l+m}{3}\PY{p}{,} \PY{l+m}{3}\PY{p}{)}
\PY{n+nf}{print}\PY{p}{(}\PY{n}{m.zeros}\PY{p}{)}
\end{Verbatim}
\end{tcolorbox}

    \begin{Verbatim}[commandchars=\\\{\}]
     [,1] [,2] [,3]
[1,]    0    1    1
[2,]    1    0    2
[3,]    2    2    0
\end{Verbatim}

    \hypertarget{a-library-of-elementary-column-and-row-operations.}{%
\subsection{A library of elementary column and row
operations.}\label{a-library-of-elementary-column-and-row-operations.}}

    \begin{tcolorbox}[breakable, size=fbox, boxrule=1pt, pad at break*=1mm,colback=cellbackground, colframe=cellborder]
\prompt{In}{incolor}{19}{\hspace{4pt}}
\begin{Verbatim}[commandchars=\\\{\}]
\PY{c+c1}{\PYZsh{}multiplies r\PYZhy{}th row of M by factor}
\PY{n}{matrix.row.divide} \PY{o}{\PYZlt{}\PYZhy{}} \PY{n+nf}{function}\PY{p}{(} \PY{n}{M}\PY{p}{,} \PY{n}{r}\PY{p}{,} \PY{n}{factor}\PY{p}{)} \PY{p}{\PYZob{}}
    \PY{n}{M}\PY{n}{[r}\PY{p}{,}\PY{n}{]} \PY{o}{=} \PY{n}{M}\PY{n}{[r}\PY{p}{,}\PY{n}{]} \PY{o}{*} \PY{n}{factor}
    \PY{n}{M} \PY{c+c1}{\PYZsh{}return}
\PY{p}{\PYZcb{}}
\end{Verbatim}
\end{tcolorbox}

    \begin{tcolorbox}[breakable, size=fbox, boxrule=1pt, pad at break*=1mm,colback=cellbackground, colframe=cellborder]
\prompt{In}{incolor}{20}{\hspace{4pt}}
\begin{Verbatim}[commandchars=\\\{\}]
\PY{c+c1}{\PYZsh{}adds to i\PYZhy{}th row tje j\PYZhy{}th row multiplied by factor}
\PY{n}{matrix.add.raw} \PY{o}{\PYZlt{}\PYZhy{}} \PY{n+nf}{function}\PY{p}{(}\PY{n}{i}\PY{p}{,} \PY{n}{j}\PY{p}{,} \PY{n}{factor}\PY{p}{,} \PY{n}{M}\PY{p}{)} \PY{p}{\PYZob{}}
    \PY{n}{M}\PY{n}{[i}\PY{p}{,}\PY{n}{]} \PY{o}{\PYZlt{}\PYZhy{}} \PY{n}{M}\PY{n}{[i}\PY{p}{,}\PY{n}{]} \PY{o}{+} \PY{n}{M}\PY{n}{[j}\PY{p}{,}\PY{n}{]} \PY{o}{*} \PY{n}{factor}
    \PY{n}{M} \PY{c+c1}{\PYZsh{} return}
\PY{p}{\PYZcb{}}
\end{Verbatim}
\end{tcolorbox}

    \begin{tcolorbox}[breakable, size=fbox, boxrule=1pt, pad at break*=1mm,colback=cellbackground, colframe=cellborder]
\prompt{In}{incolor}{21}{\hspace{4pt}}
\begin{Verbatim}[commandchars=\\\{\}]
\PY{c+c1}{\PYZsh{} swap the i\PYZhy{}th row with the j\PYZhy{}th raw}
\PY{n}{matrix.swap.raw} \PY{o}{\PYZlt{}\PYZhy{}} \PY{n+nf}{function}\PY{p}{(}\PY{n}{i}\PY{p}{,} \PY{n}{j}\PY{p}{,} \PY{n}{M}\PY{p}{)} \PY{p}{\PYZob{}}
    \PY{n}{tmp} \PY{o}{\PYZlt{}\PYZhy{}} \PY{n}{M}\PY{n}{[i}\PY{p}{,}\PY{n}{]}
    \PY{n}{M}\PY{n}{[i}\PY{p}{,}\PY{n}{]} \PY{o}{\PYZlt{}\PYZhy{}} \PY{n}{M}\PY{n}{[j}\PY{p}{,}\PY{n}{]}
    \PY{n}{M}\PY{n}{[j}\PY{p}{,}\PY{n}{]} \PY{o}{\PYZlt{}\PYZhy{}} \PY{n}{tmp}
    \PY{n}{M}
\PY{p}{\PYZcb{}}
\end{Verbatim}
\end{tcolorbox}

    \begin{tcolorbox}[breakable, size=fbox, boxrule=1pt, pad at break*=1mm,colback=cellbackground, colframe=cellborder]
\prompt{In}{incolor}{22}{\hspace{4pt}}
\begin{Verbatim}[commandchars=\\\{\}]
\PY{n}{m.swaped} \PY{o}{\PYZlt{}\PYZhy{}} \PY{n+nf}{matrix.swap.raw}\PY{p}{(}\PY{l+m}{1}\PY{p}{,}\PY{l+m}{2}\PY{p}{,}\PY{n}{m}\PY{p}{)}
\PY{n}{m}
\PY{n}{m.swaped}
\end{Verbatim}
\end{tcolorbox}

    \begin{tabular}{lll}
	 3 & 6 & 3\\
	 1 & 4 & 4\\
	 2 & 4 & 7\\
\end{tabular}


    
    \begin{tabular}{lll}
	 1 & 4 & 4\\
	 3 & 6 & 3\\
	 2 & 4 & 7\\
\end{tabular}


    
    \begin{tcolorbox}[breakable, size=fbox, boxrule=1pt, pad at break*=1mm,colback=cellbackground, colframe=cellborder]
\prompt{In}{incolor}{23}{\hspace{4pt}}
\begin{Verbatim}[commandchars=\\\{\}]
\PY{c+c1}{\PYZsh{}general Gauss row reduction algorithm}
\PY{n}{matrix.eliminate} \PY{o}{\PYZlt{}\PYZhy{}} \PY{n+nf}{function}\PY{p}{(}\PY{n}{m}\PY{p}{)}\PY{p}{\PYZob{}}
    \PY{n}{border} \PY{o}{=} \PY{n+nf}{min}\PY{p}{(}\PY{n+nf}{c}\PY{p}{(}\PY{n+nf}{nrow}\PY{p}{(}\PY{n}{m}\PY{p}{)}\PY{p}{,} \PY{n+nf}{ncol}\PY{p}{(}\PY{n}{m}\PY{p}{)}\PY{p}{)}\PY{p}{)}
    \PY{n+nf}{for }\PY{p}{(}\PY{n}{ic} \PY{n}{in} \PY{l+m}{1}\PY{o}{:}\PY{n}{border}\PY{p}{)}\PY{p}{\PYZob{}}
        \PY{c+c1}{\PYZsh{} find the index of max absolute value in column and swap rows}
        \PY{n}{to.swap} \PY{o}{\PYZlt{}\PYZhy{}} \PY{n+nf}{which}\PY{p}{(}\PY{n}{m}\PY{n}{[ic}\PY{o}{:}\PY{n+nf}{nrow}\PY{p}{(}\PY{n}{m}\PY{p}{)}\PY{p}{,}\PY{n}{ic}\PY{n}{]} \PY{o}{==} \PY{n+nf}{max}\PY{p}{(}\PY{n+nf}{abs}\PY{p}{(}\PY{n}{m}\PY{n}{[ic}\PY{o}{:}\PY{n+nf}{nrow}\PY{p}{(}\PY{n}{m}\PY{p}{)}\PY{p}{,}\PY{n}{ic}\PY{n}{]}\PY{p}{)}\PY{p}{)}\PY{p}{)} \PY{o}{+} \PY{n}{ic} \PY{o}{\PYZhy{}} \PY{l+m}{1}
        \PY{n}{m} \PY{o}{\PYZlt{}\PYZhy{}} \PY{n+nf}{matrix.swap.raw}\PY{p}{(}\PY{n}{ic}\PY{p}{,} \PY{n}{to.swap}\PY{n}{[1}\PY{n}{]}\PY{p}{,} \PY{n}{m}\PY{p}{)}
        \PY{c+c1}{\PYZsh{} makes one on the diagonal}
        \PY{n}{m} \PY{o}{\PYZlt{}\PYZhy{}} \PY{n+nf}{matrix.row.divide}\PY{p}{(}\PY{n}{m}\PY{p}{,} \PY{n}{ic}\PY{p}{,} \PY{l+m}{1}\PY{o}{/}\PY{n}{m}\PY{n}{[ic}\PY{p}{,}\PY{n}{ic}\PY{n}{]}\PY{p}{)}
        \PY{n+nf}{if }\PY{p}{(}\PY{n}{ic} \PY{o}{+} \PY{l+m}{1} \PY{o}{\PYZgt{}} \PY{n+nf}{nrow}\PY{p}{(}\PY{n}{m}\PY{p}{)}\PY{p}{)} \PY{n}{break}
        \PY{n+nf}{for }\PY{p}{(}\PY{n}{ir} \PY{n+nf}{in }\PY{p}{(}\PY{n}{ic}\PY{l+m}{+1}\PY{p}{)}\PY{o}{:}\PY{n+nf}{nrow}\PY{p}{(}\PY{n}{m}\PY{p}{)}\PY{p}{)}\PY{p}{\PYZob{}}
            \PY{n}{m} \PY{o}{\PYZlt{}\PYZhy{}} \PY{n+nf}{matrix.add.raw}\PY{p}{(}\PY{n}{ir}\PY{p}{,} \PY{n}{ic}\PY{p}{,}\PY{o}{\PYZhy{}}\PY{n}{m}\PY{n}{[ir}\PY{p}{,}\PY{n}{ic}\PY{n}{]}\PY{p}{,} \PY{n}{m}\PY{p}{)}    
        \PY{p}{\PYZcb{}}        
    \PY{p}{\PYZcb{}}
    \PY{n}{m}
\PY{p}{\PYZcb{}}
\end{Verbatim}
\end{tcolorbox}

    \hypertarget{explain-gaussian-method-of-computing-an-inverse-of-a-matrix.}{%
\subsection{Explain Gaussian method of computing an inverse of a
matrix.}\label{explain-gaussian-method-of-computing-an-inverse-of-a-matrix.}}

    \begin{tcolorbox}[breakable, size=fbox, boxrule=1pt, pad at break*=1mm,colback=cellbackground, colframe=cellborder]
\prompt{In}{incolor}{24}{\hspace{4pt}}
\begin{Verbatim}[commandchars=\\\{\}]
\PY{n}{matrix.solve} \PY{o}{\PYZlt{}\PYZhy{}} \PY{n+nf}{function}\PY{p}{(}\PY{n}{m}\PY{p}{)}\PY{p}{\PYZob{}}
    \PY{c+c1}{\PYZsh{} check if invertible}
    \PY{n+nf}{if }\PY{p}{(}\PY{n+nf}{det}\PY{p}{(}\PY{n}{m}\PY{p}{)} \PY{o}{==} \PY{l+m}{0}\PY{p}{)}\PY{p}{\PYZob{}}
        \PY{l+m}{0} 
    \PY{p}{\PYZcb{}} \PY{n}{else} \PY{p}{\PYZob{}}
    \PY{c+c1}{\PYZsh{} append identity matrix to m}
    \PY{n}{mI} \PY{o}{\PYZlt{}\PYZhy{}} \PY{n+nf}{cbind}\PY{p}{(}\PY{n}{m}\PY{p}{,} \PY{n+nf}{diag}\PY{p}{(}\PY{n+nf}{nrow}\PY{p}{(}\PY{n}{m}\PY{p}{)}\PY{p}{)}\PY{p}{)}
    \PY{c+c1}{\PYZsh{} gaussion elimination we get ones on diagonal}
    \PY{n}{mI} \PY{o}{\PYZlt{}\PYZhy{}} \PY{n+nf}{matrix.eliminate}\PY{p}{(}\PY{n}{mI}\PY{p}{)}
    \PY{n+nf}{for }\PY{p}{(}\PY{n}{i} \PY{n}{in} \PY{n+nf}{nrow}\PY{p}{(}\PY{n}{mI}\PY{p}{)}\PY{o}{:}\PY{l+m}{2}\PY{p}{)} \PY{p}{\PYZob{}}
        \PY{n+nf}{for }\PY{p}{(}\PY{n}{j} \PY{n+nf}{in }\PY{p}{(}\PY{n}{i}\PY{l+m}{\PYZhy{}1}\PY{p}{)}\PY{o}{:}\PY{l+m}{1}\PY{p}{)} \PY{p}{\PYZob{}}
            \PY{n}{mI} \PY{o}{\PYZlt{}\PYZhy{}} \PY{n+nf}{matrix.add.raw}\PY{p}{(}\PY{n}{j}\PY{p}{,} \PY{n}{i}\PY{p}{,} \PY{o}{\PYZhy{}}\PY{n}{mI}\PY{n}{[j}\PY{p}{,}\PY{n}{i}\PY{n}{]}\PY{p}{,} \PY{n}{mI}\PY{p}{)}    
        \PY{p}{\PYZcb{}}
    \PY{p}{\PYZcb{}}
    \PY{n}{mI}\PY{n}{[}\PY{p}{,}\PY{p}{(}\PY{n+nf}{nrow}\PY{p}{(}\PY{n}{m}\PY{p}{)} \PY{o}{+} \PY{l+m}{1}\PY{p}{)}\PY{o}{:}\PY{n+nf}{ncol}\PY{p}{(}\PY{n}{mI}\PY{p}{)}\PY{n}{]}
    \PY{p}{\PYZcb{}}
\PY{p}{\PYZcb{}}
\end{Verbatim}
\end{tcolorbox}

    \begin{tcolorbox}[breakable, size=fbox, boxrule=1pt, pad at break*=1mm,colback=cellbackground, colframe=cellborder]
\prompt{In}{incolor}{25}{\hspace{4pt}}
\begin{Verbatim}[commandchars=\\\{\}]
\PY{n+nf}{print}\PY{p}{(}\PY{l+s}{\PYZdq{}}\PY{l+s}{m original\PYZdq{}}\PY{p}{)}
\PY{n}{m}
\PY{n}{me} \PY{o}{\PYZlt{}\PYZhy{}} \PY{n+nf}{matrix.eliminate}\PY{p}{(}\PY{n}{m}\PY{p}{)}
\PY{n}{me}

\PY{n+nf}{print}\PY{p}{(}\PY{l+s}{\PYZdq{}}\PY{l+s}{m1 original\PYZdq{}}\PY{p}{)}
\PY{n}{m1}
\PY{n}{m1e} \PY{o}{\PYZlt{}\PYZhy{}} \PY{n+nf}{matrix.eliminate}\PY{p}{(}\PY{n}{m1}\PY{p}{)}
\PY{n}{m1e}
\PY{n+nf}{print}\PY{p}{(}\PY{l+s}{\PYZdq{}}\PY{l+s}{m2 original\PYZdq{}}\PY{p}{)}
\PY{n}{m2}
\PY{n}{m2e} \PY{o}{\PYZlt{}\PYZhy{}} \PY{n+nf}{matrix.eliminate}\PY{p}{(}\PY{n}{m2}\PY{p}{)}
\PY{n}{m2e}

\PY{n+nf}{print}\PY{p}{(}\PY{l+s}{\PYZdq{}}\PY{l+s}{m.zeros original\PYZdq{}}\PY{p}{)}
\PY{n}{m.zeros}
\PY{n}{m.zeros.e} \PY{o}{\PYZlt{}\PYZhy{}} \PY{n+nf}{matrix.eliminate}\PY{p}{(}\PY{n}{m.zeros}\PY{p}{)}
\PY{n}{m.zeros.e}
\end{Verbatim}
\end{tcolorbox}

    \begin{Verbatim}[commandchars=\\\{\}]
[1] "m original"
\end{Verbatim}

    \begin{tabular}{lll}
	 3 & 6 & 3\\
	 1 & 4 & 4\\
	 2 & 4 & 7\\
\end{tabular}


    
    \begin{tabular}{lll}
	 1   & 2   & 1.0\\
	 0   & 1   & 1.5\\
	 0   & 0   & 1.0\\
\end{tabular}


    
    \begin{Verbatim}[commandchars=\\\{\}]
[1] "m1 original"
\end{Verbatim}

    \begin{tabular}{llll}
	 3 & 6 & 3 & 1\\
	 1 & 4 & 4 & 2\\
	 2 & 4 & 7 & 3\\
\end{tabular}


    
    \begin{tabular}{llll}
	 1         & 2         & 1.0       & 0.3333333\\
	 0         & 1         & 1.5       & 0.8333333\\
	 0         & 0         & 1.0       & 0.4666667\\
\end{tabular}


    
    \begin{Verbatim}[commandchars=\\\{\}]
[1] "m2 original"
\end{Verbatim}

    \begin{tabular}{lll}
	 3 & 4 & 7\\
	 1 & 4 & 1\\
	 2 & 3 & 2\\
	 6 & 4 & 3\\
\end{tabular}


    
    \begin{tabular}{lll}
	 1         & 0.6666667 & 0.50     \\
	 0         & 1.0000000 & 0.15     \\
	 0         & 0.0000000 & 1.00     \\
	 0         & 0.0000000 & 0.00     \\
\end{tabular}


    
    \begin{Verbatim}[commandchars=\\\{\}]
[1] "m.zeros original"
\end{Verbatim}

    \begin{tabular}{lll}
	 0 & 1 & 1\\
	 1 & 0 & 2\\
	 2 & 2 & 0\\
\end{tabular}


    
    \begin{tabular}{lll}
	 1 & 1 & 0\\
	 0 & 1 & 1\\
	 0 & 0 & 1\\
\end{tabular}


    
    \begin{tcolorbox}[breakable, size=fbox, boxrule=1pt, pad at break*=1mm,colback=cellbackground, colframe=cellborder]
\prompt{In}{incolor}{26}{\hspace{4pt}}
\begin{Verbatim}[commandchars=\\\{\}]
\PY{n}{m}
\PY{n+nf}{matrix.solve}\PY{p}{(}\PY{n}{m}\PY{p}{)}
\PY{n+nf}{solve}\PY{p}{(}\PY{n}{m}\PY{p}{)}
\end{Verbatim}
\end{tcolorbox}

    \begin{tabular}{lll}
	 3 & 6 & 3\\
	 1 & 4 & 4\\
	 2 & 4 & 7\\
\end{tabular}


    
    \begin{tabular}{lll}
	  0.40000000 & -1.0        &  0.4       \\
	  0.03333333 &  0.5        & -0.3       \\
	 -0.13333333 &  0.0        &  0.2       \\
\end{tabular}


    
    \begin{tabular}{lll}
	  0.40000000 & -1.0        &  0.4       \\
	  0.03333333 &  0.5        & -0.3       \\
	 -0.13333333 &  0.0        &  0.2       \\
\end{tabular}


    
    \hypertarget{explain-gaussian-method-of-computing-the-rank-of-a-matrix.}{%
\subsection{Explain Gaussian method of computing the rank of a
matrix.}\label{explain-gaussian-method-of-computing-the-rank-of-a-matrix.}}

    \begin{tcolorbox}[breakable, size=fbox, boxrule=1pt, pad at break*=1mm,colback=cellbackground, colframe=cellborder]
\prompt{In}{incolor}{27}{\hspace{4pt}}
\begin{Verbatim}[commandchars=\\\{\}]
\PY{n}{matrix.rank} \PY{o}{\PYZlt{}\PYZhy{}} \PY{n+nf}{function}\PY{p}{(}\PY{n}{m}\PY{p}{)}\PY{p}{\PYZob{}}
    \PY{n}{border} \PY{o}{=} \PY{n+nf}{min}\PY{p}{(}\PY{n+nf}{c}\PY{p}{(}\PY{n+nf}{nrow}\PY{p}{(}\PY{n}{m}\PY{p}{)}\PY{p}{,} \PY{n+nf}{ncol}\PY{p}{(}\PY{n}{m}\PY{p}{)}\PY{p}{)}\PY{p}{)}
    \PY{n+nf}{if }\PY{p}{(}\PY{n+nf}{ncol}\PY{p}{(}\PY{n}{m}\PY{p}{)} \PY{o}{\PYZgt{}} \PY{n+nf}{nrow}\PY{p}{(}\PY{n}{m}\PY{p}{)}\PY{p}{)} \PY{p}{\PYZob{}}
        \PY{n}{m} \PY{o}{\PYZlt{}\PYZhy{}} \PY{n+nf}{t}\PY{p}{(}\PY{n}{m}\PY{p}{)}
    \PY{p}{\PYZcb{}}
    \PY{n}{rank} \PY{o}{\PYZlt{}\PYZhy{}} \PY{n}{border}
    \PY{n+nf}{for }\PY{p}{(}\PY{n}{ic} \PY{n}{in} \PY{l+m}{1}\PY{o}{:}\PY{n}{border}\PY{p}{)}\PY{p}{\PYZob{}}
        \PY{n+nf}{if }\PY{p}{(}\PY{n}{m}\PY{n}{[ic}\PY{p}{,} \PY{n}{ic}\PY{n}{]} \PY{o}{==} \PY{l+m}{0}\PY{p}{)}\PY{p}{\PYZob{}}
            \PY{n}{non.zero.index} \PY{o}{\PYZlt{}\PYZhy{}} \PY{n+nf}{which}\PY{p}{(}\PY{n}{m}\PY{n}{[ic}\PY{o}{:}\PY{n+nf}{nrow}\PY{p}{(}\PY{n}{m}\PY{p}{)}\PY{p}{,} \PY{n}{ic}\PY{n}{]} \PY{o}{!=} \PY{l+m}{0}\PY{p}{)} \PY{o}{+} \PY{n}{ic} \PY{l+m}{\PYZhy{}1}
            \PY{c+c1}{\PYZsh{} if exists non zero value in the column}
            \PY{n+nf}{if}\PY{p}{(}\PY{n+nf}{length}\PY{p}{(}\PY{n}{non.zero.index}\PY{p}{)} \PY{o}{\PYZgt{}} \PY{l+m}{0}\PY{p}{)} \PY{p}{\PYZob{}}
                \PY{n}{m} \PY{o}{\PYZlt{}\PYZhy{}} \PY{n+nf}{matrix.swap.raw}\PY{p}{(}\PY{n}{non.zero.index}\PY{n}{[1}\PY{n}{]}\PY{p}{,} \PY{n}{ic}\PY{p}{,} \PY{n}{m}\PY{p}{)}
            \PY{p}{\PYZcb{}} \PY{n}{else} \PY{p}{\PYZob{}}
                \PY{c+c1}{\PYZsh{} there is no value different than zero}
                \PY{c+c1}{\PYZsh{} whole column is 0}
                \PY{n}{rank} \PY{o}{\PYZlt{}\PYZhy{}} \PY{n}{rank} \PY{o}{\PYZhy{}} \PY{l+m}{1}
                \PY{n}{next}
            \PY{p}{\PYZcb{}}
        \PY{p}{\PYZcb{}}
        \PY{n+nf}{for }\PY{p}{(}\PY{n}{ir} \PY{n}{in} \PY{l+m}{1}\PY{o}{:}\PY{n+nf}{nrow}\PY{p}{(}\PY{n}{m}\PY{p}{)}\PY{p}{)}\PY{p}{\PYZob{}}
            \PY{n+nf}{if }\PY{p}{(}\PY{n}{ir} \PY{o}{!=} \PY{n}{ic}\PY{p}{)}\PY{p}{\PYZob{}}
                \PY{n}{m} \PY{o}{\PYZlt{}\PYZhy{}} \PY{n+nf}{matrix.add.raw}\PY{p}{(}\PY{n}{ir}\PY{p}{,} \PY{n}{ic}\PY{p}{,} \PY{o}{\PYZhy{}}\PY{n}{m}\PY{n}{[ir}\PY{p}{,} \PY{n}{ic}\PY{n}{]}\PY{o}{/}\PY{n}{m}\PY{n}{[ic}\PY{p}{,}\PY{n}{ic}\PY{n}{]}\PY{p}{,} \PY{n}{m}\PY{p}{)}
            \PY{p}{\PYZcb{}}
        \PY{p}{\PYZcb{}}
    \PY{p}{\PYZcb{}}
    \PY{n}{rank}
\PY{p}{\PYZcb{}}
\end{Verbatim}
\end{tcolorbox}

    \begin{tcolorbox}[breakable, size=fbox, boxrule=1pt, pad at break*=1mm,colback=cellbackground, colframe=cellborder]
\prompt{In}{incolor}{28}{\hspace{4pt}}
\begin{Verbatim}[commandchars=\\\{\}]
\PY{n}{m} \PY{o}{\PYZlt{}\PYZhy{}} \PY{n+nf}{matrix}\PY{p}{(}\PY{n+nf}{c}\PY{p}{(}\PY{l+m}{0}\PY{p}{,}\PY{l+m}{0}\PY{p}{,}\PY{l+m}{0}\PY{p}{,}\PY{l+m}{1}\PY{p}{,}\PY{l+m}{2}\PY{p}{,}\PY{l+m}{3}\PY{p}{,}\PY{l+m}{2}\PY{p}{,}\PY{l+m}{3}\PY{p}{,}\PY{l+m}{0}\PY{p}{,}\PY{l+m}{1}\PY{p}{,}\PY{l+m}{1}\PY{p}{,}\PY{l+m}{1}\PY{p}{,}\PY{l+m}{10}\PY{p}{,}\PY{l+m}{8}\PY{p}{,}\PY{l+m}{5}\PY{p}{,}\PY{l+m}{3}\PY{p}{,}\PY{l+m}{4}\PY{p}{,}\PY{l+m}{5}\PY{p}{)}\PY{p}{,}\PY{l+m}{6}\PY{p}{)}
\PY{n}{m}
\PY{n+nf}{matrix.rank}\PY{p}{(}\PY{n}{m}\PY{p}{)}
\PY{n+nf}{qr}\PY{p}{(}\PY{n}{m}\PY{p}{)}\PY{o}{\PYZdl{}}\PY{n}{rank}
\PY{n+nf}{matrix.rank}\PY{p}{(}\PY{n}{m1}\PY{p}{)} \PY{o}{==} \PY{n+nf}{qr}\PY{p}{(}\PY{n}{m1}\PY{p}{)}\PY{o}{\PYZdl{}}\PY{n}{rank}
\PY{n+nf}{matrix.rank}\PY{p}{(}\PY{n}{m2}\PY{p}{)} \PY{o}{==} \PY{n+nf}{qr}\PY{p}{(}\PY{n}{m2}\PY{p}{)}\PY{o}{\PYZdl{}}\PY{n}{rank}
\PY{n+nf}{matrix.rank}\PY{p}{(}\PY{n}{m.zeros}\PY{p}{)} \PY{o}{==} \PY{n+nf}{qr}\PY{p}{(}\PY{n}{m.zeros}\PY{p}{)}\PY{o}{\PYZdl{}}\PY{n}{rank}
\end{Verbatim}
\end{tcolorbox}

    \begin{tabular}{lll}
	 0  & 2  & 10\\
	 0  & 3  &  8\\
	 0  & 0  &  5\\
	 1  & 1  &  3\\
	 2  & 1  &  4\\
	 3  & 1  &  5\\
\end{tabular}


    
    3

    
    3

    
    TRUE

    
    TRUE

    
    TRUE

    
    \hypertarget{propose-a-method-of-calculation-of-a-dimension-of-linear-span-spana-of-given-finite-set-of-vectors-arn}{%
\subsection{\texorpdfstring{Propose a method of calculation of a
dimension of linear span \(Span(A)\) of given finite set of vectors
\(A⊆Rn\)}{Propose a method of calculation of a dimension of linear span Span(A) of given finite set of vectors A⊆Rn}}\label{propose-a-method-of-calculation-of-a-dimension-of-linear-span-spana-of-given-finite-set-of-vectors-arn}}

    We need to find all lineary independent vectors from A, which will be
the base of \(Span(A)\). Cardinality of this base will be dimension of
\(Span(A)\). Vectors must be lineary independent not just non collinear.
Number of lineary independent vectors from \(A\) is the definiotion of
\(rank(A)\). We just need to compute rank on the given matrix \(A\).


    % Add a bibliography block to the postdoc
    
    
    
    \end{document}
